\documentclass[a4paper]{exam}

\usepackage{amsmath, amssymb, amsthm, amsfonts}
\usepackage[utf8]{inputenc}
\usepackage{vntex}

%Page setup
\usepackage[letterpaper, top = 1.0in, bottom = 1.0in, left = 1.0in, right = 1.0in, heightrounded]{geometry}

\headrule
\header{\textbf{Biên soạn: Trần Minh Đức}}{}{\textbf{7/3/2025}}

\begin{document}
	\begin{center}
		\fbox{\parbox{5.5in}{\centering
				\vspace{1mm}
				HCMUS - TOÁN RỜI RẠC (CNTT) - 13/11/2024\\
				\vspace{1mm}
				HỌC KÌ I - NĂM HỌC 2024 - 2025 - THỜI GIAN: 60 PHÚT\\
				\vspace{1mm}
			}
		}
	\end{center}
	
	\vspace*{2mm}
	
	\begin{questions}
		\question (\textbf{3.5 điểm} = 1đ + 1đ + 0.5đ + 1đ). \textbf{Cho các biến mệnh đề p, q, r, s và t.}
		\begin{itemize}
			\item Đặt $A = \left[((p \longrightarrow r) \land q) \longrightarrow (p \land q) \right]$ và $B = (\neg p \longrightarrow \neg q)$. Chứng minh $A \longleftrightarrow B$.\\
			\\Nếu p đúng thì chân trị của A ra sao? (Dùng B để giải thích ngắn gọn).\\
			\item Xét các suy luận sau:\\
			\[
			\begin{array}{c|c}
				\textbf{Suy luận bên trái} & \textbf{Suy luận bên phải}\\
				\hline
				\neg s \land t \quad (1) & r \lor t \quad (6)\\[1mm]
				p \lor q \quad (2) & \neg p \longrightarrow (q \land s) \quad (7)\\[1mm]
				q \longrightarrow (r \longrightarrow s) \quad (3) & p \land \neg t \quad (8)\\[1mm]
				t \longrightarrow q \quad (4) & r \longrightarrow \neg q \quad (9)\\[1mm]
				\hline
				\therefore p \quad (5) & \therefore \neg q \longrightarrow s \quad (10)
			\end{array}
			\]\\
			Hãy chứng minh suy luận bên trái là \textit{đúng} và giải thích tại sao suy luận bên phải là \textit{sai}.\\
			\item Cho $C = "\forall x \in \mathbb{R}, \exists y \in \mathbf{Q}, y = sin(3x)$ hay $y = cos(x)$. Viết mệnh đề phủ định $\neg C$ và xét chân trị của C.\\
		\end{itemize}
		\question (\textbf{3 điểm} = 1đ + 2đ)
		\begin{itemize}
			\item Cho $ A, B, C, D \subset E$. Chứng minh $\left[A \setminus (B \cup C) \right] \, \cup \, \left[(A \setminus B) \cap D \right] \, = \, \left[(A \setminus B) \setminus (C \cap \neg D)\right]$.\\
			\\Nếu $D = \emptyset$ thì hãy rút gọn $\left[(A \setminus B) \cap D \right]$ và $(C \cap \neg D)$.\\
			\item Cho $f: \mathbb{R} \longrightarrow \mathbb{R}$ có $f(x) = e^{-x} - 2e^{x} + 5, \forall x \in \mathbb{R}$, $g: \mathbb{R} \setminus \{0\} \longrightarrow \mathbb{R}$ có $g(x) = 2x^{2} - x^{-2} + 5, \forall x \in \mathbb{R} \setminus \{0\}$ và $h: \mathbb{R} \setminus \{0\} \longrightarrow \mathbb{R}$ thoả $f \circ g = h$.\\
			\\Chứng minh rằng $f$ là một song ánh và tìm biểu thức của ánh xạ ngược $f^{-1}$.\\
			\\Tìm biểu thức của $h$ (yêu cầu viết dưới dạng rút gọn).\\
		\end{itemize}
		\question (\textbf{3.5 điểm} = 1.5đ + 1đ + 1đ)
		\begin{itemize}
			\item Cho $S = {0, 1, 2,..., 8, 9, 10}$. Hỏi $S$ có bao nhiêu tập hợp con?\\
			\\S có bao nhiêu tập hợp con T thoả $\|T\| = 6$, $minT = 1$ và $8 \leq maxT \leq 9$?\\
			\item Xếp a, a, a, b, b, b, c, c, c, c thành một dãy ký tự tuỳ ý có 10 ký tự (chẳng hạn dãy \textit{cabcbaacbc},...). Hỏi có tất cả bao nhiêu dãy ký tự như vậy?\\
			Nếu yêu cầu thêm ký tự đầu của dãy là a và ký tự cuối của dãy phải khác a thì ta có bao nhiêu dãy?\\
			\item Khi khai triển biểu thức $(2x - 3y^{2} + 4z^{3} - 5t^{4})^{14} $ ta được bao nhiêu đơn thức khác nhau và hệ số đứng trước $x^{8}y^{4}z^{9}t^{4}$ là bao nhiêu?\\ 
		\end{itemize}
	\end{questions}
	\hspace*{0pt}\hfill \textit{\textbf{HẾT}}
	
	\pagebreak
\end{document}